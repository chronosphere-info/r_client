\documentclass[]{article}
\usepackage{lmodern}
\usepackage{amssymb,amsmath}
\usepackage{ifxetex,ifluatex}
\usepackage{fixltx2e} % provides \textsubscript
\ifnum 0\ifxetex 1\fi\ifluatex 1\fi=0 % if pdftex
  \usepackage[T1]{fontenc}
  \usepackage[utf8]{inputenc}
\else % if luatex or xelatex
  \ifxetex
    \usepackage{mathspec}
  \else
    \usepackage{fontspec}
  \fi
  \defaultfontfeatures{Ligatures=TeX,Scale=MatchLowercase}
\fi
% use upquote if available, for straight quotes in verbatim environments
\IfFileExists{upquote.sty}{\usepackage{upquote}}{}
% use microtype if available
\IfFileExists{microtype.sty}{%
\usepackage{microtype}
\UseMicrotypeSet[protrusion]{basicmath} % disable protrusion for tt fonts
}{}
\usepackage[margin=1in]{geometry}
\usepackage{hyperref}
\hypersetup{unicode=true,
            pdftitle={Introduction to the `chronosphere' R package},
            pdfauthor={Adam T. Kocsis, Nussaibah B. Raja},
            pdfborder={0 0 0},
            breaklinks=true}
\urlstyle{same}  % don't use monospace font for urls
\usepackage{color}
\usepackage{fancyvrb}
\newcommand{\VerbBar}{|}
\newcommand{\VERB}{\Verb[commandchars=\\\{\}]}
\DefineVerbatimEnvironment{Highlighting}{Verbatim}{commandchars=\\\{\}}
% Add ',fontsize=\small' for more characters per line
\usepackage{framed}
\definecolor{shadecolor}{RGB}{248,248,248}
\newenvironment{Shaded}{\begin{snugshade}}{\end{snugshade}}
\newcommand{\AlertTok}[1]{\textcolor[rgb]{0.94,0.16,0.16}{#1}}
\newcommand{\AnnotationTok}[1]{\textcolor[rgb]{0.56,0.35,0.01}{\textbf{\textit{#1}}}}
\newcommand{\AttributeTok}[1]{\textcolor[rgb]{0.77,0.63,0.00}{#1}}
\newcommand{\BaseNTok}[1]{\textcolor[rgb]{0.00,0.00,0.81}{#1}}
\newcommand{\BuiltInTok}[1]{#1}
\newcommand{\CharTok}[1]{\textcolor[rgb]{0.31,0.60,0.02}{#1}}
\newcommand{\CommentTok}[1]{\textcolor[rgb]{0.56,0.35,0.01}{\textit{#1}}}
\newcommand{\CommentVarTok}[1]{\textcolor[rgb]{0.56,0.35,0.01}{\textbf{\textit{#1}}}}
\newcommand{\ConstantTok}[1]{\textcolor[rgb]{0.00,0.00,0.00}{#1}}
\newcommand{\ControlFlowTok}[1]{\textcolor[rgb]{0.13,0.29,0.53}{\textbf{#1}}}
\newcommand{\DataTypeTok}[1]{\textcolor[rgb]{0.13,0.29,0.53}{#1}}
\newcommand{\DecValTok}[1]{\textcolor[rgb]{0.00,0.00,0.81}{#1}}
\newcommand{\DocumentationTok}[1]{\textcolor[rgb]{0.56,0.35,0.01}{\textbf{\textit{#1}}}}
\newcommand{\ErrorTok}[1]{\textcolor[rgb]{0.64,0.00,0.00}{\textbf{#1}}}
\newcommand{\ExtensionTok}[1]{#1}
\newcommand{\FloatTok}[1]{\textcolor[rgb]{0.00,0.00,0.81}{#1}}
\newcommand{\FunctionTok}[1]{\textcolor[rgb]{0.00,0.00,0.00}{#1}}
\newcommand{\ImportTok}[1]{#1}
\newcommand{\InformationTok}[1]{\textcolor[rgb]{0.56,0.35,0.01}{\textbf{\textit{#1}}}}
\newcommand{\KeywordTok}[1]{\textcolor[rgb]{0.13,0.29,0.53}{\textbf{#1}}}
\newcommand{\NormalTok}[1]{#1}
\newcommand{\OperatorTok}[1]{\textcolor[rgb]{0.81,0.36,0.00}{\textbf{#1}}}
\newcommand{\OtherTok}[1]{\textcolor[rgb]{0.56,0.35,0.01}{#1}}
\newcommand{\PreprocessorTok}[1]{\textcolor[rgb]{0.56,0.35,0.01}{\textit{#1}}}
\newcommand{\RegionMarkerTok}[1]{#1}
\newcommand{\SpecialCharTok}[1]{\textcolor[rgb]{0.00,0.00,0.00}{#1}}
\newcommand{\SpecialStringTok}[1]{\textcolor[rgb]{0.31,0.60,0.02}{#1}}
\newcommand{\StringTok}[1]{\textcolor[rgb]{0.31,0.60,0.02}{#1}}
\newcommand{\VariableTok}[1]{\textcolor[rgb]{0.00,0.00,0.00}{#1}}
\newcommand{\VerbatimStringTok}[1]{\textcolor[rgb]{0.31,0.60,0.02}{#1}}
\newcommand{\WarningTok}[1]{\textcolor[rgb]{0.56,0.35,0.01}{\textbf{\textit{#1}}}}
\usepackage{graphicx,grffile}
\makeatletter
\def\maxwidth{\ifdim\Gin@nat@width>\linewidth\linewidth\else\Gin@nat@width\fi}
\def\maxheight{\ifdim\Gin@nat@height>\textheight\textheight\else\Gin@nat@height\fi}
\makeatother
% Scale images if necessary, so that they will not overflow the page
% margins by default, and it is still possible to overwrite the defaults
% using explicit options in \includegraphics[width, height, ...]{}
\setkeys{Gin}{width=\maxwidth,height=\maxheight,keepaspectratio}
\IfFileExists{parskip.sty}{%
\usepackage{parskip}
}{% else
\setlength{\parindent}{0pt}
\setlength{\parskip}{6pt plus 2pt minus 1pt}
}
\setlength{\emergencystretch}{3em}  % prevent overfull lines
\providecommand{\tightlist}{%
  \setlength{\itemsep}{0pt}\setlength{\parskip}{0pt}}
\setcounter{secnumdepth}{0}
% Redefines (sub)paragraphs to behave more like sections
\ifx\paragraph\undefined\else
\let\oldparagraph\paragraph
\renewcommand{\paragraph}[1]{\oldparagraph{#1}\mbox{}}
\fi
\ifx\subparagraph\undefined\else
\let\oldsubparagraph\subparagraph
\renewcommand{\subparagraph}[1]{\oldsubparagraph{#1}\mbox{}}
\fi

%%% Use protect on footnotes to avoid problems with footnotes in titles
\let\rmarkdownfootnote\footnote%
\def\footnote{\protect\rmarkdownfootnote}

%%% Change title format to be more compact
\usepackage{titling}

% Create subtitle command for use in maketitle
\providecommand{\subtitle}[1]{
  \posttitle{
    \begin{center}\large#1\end{center}
    }
}

\setlength{\droptitle}{-2em}

  \title{Introduction to the `chronosphere' R package}
    \pretitle{\vspace{\droptitle}\centering\huge}
  \posttitle{\par}
    \author{Adam T. Kocsis, Nussaibah B. Raja}
    \preauthor{\centering\large\emph}
  \postauthor{\par}
      \predate{\centering\large\emph}
  \postdate{\par}
    \date{2019-11-28}


\begin{document}
\maketitle

\hypertarget{introduction}{%
\section{1. Introduction}\label{introduction}}

\hypertarget{installation}{%
\subsection{1.1. Installation}\label{installation}}

To install this alpha version of the package, you must download it
either from the CRAN servers or its dedicated GitHub repository
(\url{http://www.github.com/adamkocsis/chronosphere/}). All minor
updates will be posted on GitHub as soon as they are finished, so please
check this regularly. The version on CRAN will be lagging for some time,
as it takes the servers many days to process everything. All questions
and bugs can be reported at the github issues board
(\url{https://github.com/adamkocsis/chronosphere/issues}). Instead of
spending it on actual research, a tremendous amount of time was invested
in making this piece of software streamlined and user-friendly. If you
use a dataset of the package in a publication, please cite both its
reference(s) and the chronosphere' package itself.

After installing, from the CRAN, from a source or with
\texttt{devtools::install\_github()} , you can attach the package with:

\begin{Shaded}
\begin{Highlighting}[]
\KeywordTok{library}\NormalTok{(chronosphere)}
\end{Highlighting}
\end{Shaded}

\hypertarget{general-features}{%
\section{1.2 General features}\label{general-features}}

The purpose of the `chronosphere' project is to facilitate, streamline
and fasten the finding, acquisition and importing of Earth science data
in R. Although the package currently focuses on deep time datasets, the
scope of the included datasets will be much larger, spanning from a
single variable published as supplementary material in a journal
article, to GIS data or the entire output of GCM models. `chronosphere'
intends to decrease the gap between research hypotheses and the finding,
download and importing of datasets.

\hypertarget{rasterarray}{%
\section{2. RasterArray}\label{rasterarray}}

Faster data importing and better organization represents a considerable
part of this process. Spatially explicit data are excellent candidatates
to demonstrate how more efficient data organization can speed up
research. Although R has an excellent infrastructure for handling raster
data (\$hijmans), the arrangement of individual layers are rather
limited, which can be a problem, when a large number of layers have to
be considered. RasterStacks and RasterBricks are very efficient for
organizing RasterLayers according to a single dimension (e.g.~depth for
3D variables, or time), multidimensional structures are preferred.

To offer a more effective solution, the `chronosphere' package includes
the definition of the RasterArray S4 class. RasterArrays represent
hybrids between RasterStacks and regular R arrays. In short, they are
virtual arrays of RasterLayers, and can be thought of as regular arrays,
that include entire rasters as elements rather than single numeric,
logical or character values. As regular R users are familiar with
subsetting, combinations and structures of regular arrays (including
formal vectors and matrices), the finding, extraction and storage of
spatially explicit data is much easier in such containers.

\hypertarget{strucuture}{%
\subsection{2.1. Strucuture}\label{strucuture}}

RasterArrays do not directly inherit from Raster* objects of the raster
package, as a considerable number of main functions differ, but they
rather represent a wrapper object around regular RasterStacks - stacks
of individual RasterLayers. This ensures that whenever users are
unfamililar with the methods of RasterArray class objects, they can
always reduce their data to stacks or individual layers.

The formal class RasterArray has only two slots: a stack and and index.
The stack includes the Raster data in an unfolded manner, similarly to
how matrices and arrays are essentially just vectors with additional
attributes. The stack slot incorporates a single RasterStack object,
which represents the data content of the object. The index slot, on the
other hand, describes the structure of the RasterArray. It is a
vector/matrix/array of integers each representing an index of the layers
in the stack. The configuration (dimensions) of the index represents the
entire array.

The `chronosphere' package includes two demo datasets: a set of ancient
topogrophies (PaleoDEMs, SCotese..) and time series of bioclimatic
variables (annual mean temperature and precipitation) from the CHELSA
project. (\$ref) These datasets can be attached with the data command.

\begin{Shaded}
\begin{Highlighting}[]
\KeywordTok{data}\NormalTok{(dems)}
\KeywordTok{data}\NormalTok{(clim)}
\end{Highlighting}
\end{Shaded}

THe structure of RasterArrays can be inspected if the object's name is
typed into the console:

\begin{Shaded}
\begin{Highlighting}[]
\NormalTok{dems}
\end{Highlighting}
\end{Shaded}

\begin{verbatim}
## class         : RasterArray 
## RasterLayer properties: 
## - dimensions  : 181, 361  (nrow, ncol)
## - resolution  : 1, 1  (x, y)
## - extent      : -180.5, 180.5, -90.5, 90.5  (xmin, xmax, ymin, ymax)
## - coord. ref. : +proj=longlat +datum=WGS84 +ellps=WGS84 +towgs84=0,0,0 
## Array properties: 
## - dimensions   : 10  (vector)
## - num. layers    : 10
## - proxy:
##         0        5       10       15       20       25       30       35 
##  "dem_0"  "dem_5" "dem_10" "dem_15" "dem_20" "dem_25" "dem_30" "dem_35" 
##       40       45 
## "dem_40" "dem_45"
\end{verbatim}

The first part of the console output includes propoerties of the
individual RasterLayers stored in the stack. These layers have to share
essential attributes that allow them to be stored in a single stack
(extent, resolution, CRS).

The second part of the output is a visualization of the strucutre of the
RasterArray itself. In the case of the DEMs, 10 layers are stored in the
stack, each layer having its individual name (e.g.~`dem\_0'). It is a
single dimensional array (vector), and each element has its name in the
array (``0''). The differentiating between the names of layers and the
names of elements allows different subsetting and replacement rules for
the two, which both can be handy - depending on the needs of the user.

The strucuture of the RasterArray can be visualized, analyzed or
processed using the the proxy object. Proxies are essentially the same
as the index slots of the RasterArray, but instead of including the
indices of the layers they represent, proxies include the names of the
layers. These can be accessed using the proxy() function.

\begin{Shaded}
\begin{Highlighting}[]
\KeywordTok{proxy}\NormalTok{(dems)}
\end{Highlighting}
\end{Shaded}

\begin{verbatim}
##        0        5       10       15       20       25       30       35 
##  "dem_0"  "dem_5" "dem_10" "dem_15" "dem_20" "dem_25" "dem_30" "dem_35" 
##       40       45 
## "dem_40" "dem_45"
\end{verbatim}

Proxies are displayed as the second parts of the console output when the
name of the object is typed into the console (show method).

\begin{Shaded}
\begin{Highlighting}[]
\NormalTok{clim}
\end{Highlighting}
\end{Shaded}

\begin{verbatim}
## class         : RasterArray 
## RasterLayer properties: 
## - dimensions  : 181, 361  (nrow, ncol)
## - resolution  : 1, 1  (x, y)
## - extent      : -180.5, 180.5, -90.5, 90.5  (xmin, xmax, ymin, ymax)
## - coord. ref. : +proj=longlat +datum=WGS84 +ellps=WGS84 +towgs84=0,0,0 
## Array properties: 
## - dimensions  : 10, 2  (nrow, ncol)
## - num. layers    : 20
## - proxy:
##       bio1       bio12     
## 2001 "dem_0.1"  "dem_0.2" 
## 2002 "dem_5.1"  "dem_5.2" 
## 2003 "dem_10.1" "dem_10.2"
## 2004 "dem_15.1" "dem_15.2"
## 2005 "dem_20.1" "dem_20.2"
## 2006 "dem_25.1" "dem_25.2"
## 2007 "dem_30.1" "dem_30.2"
## 2008 "dem_35.1" "dem_35.2"
## 2009 "dem_40.1" "dem_40.2"
## 2010 "dem_45.1" "dem_45.2"
\end{verbatim}

This RasterArray has 10 rows (annual means) and two variables/columns:
temperature (bio1) and precipitation (bio12). With the proxy() function
it is easy to interact with this object, or to query or analyze it.

\begin{Shaded}
\begin{Highlighting}[]
\KeywordTok{proxy}\NormalTok{(clim)}
\end{Highlighting}
\end{Shaded}

\begin{verbatim}
##      bio1       bio12     
## 2001 "dem_0.1"  "dem_0.2" 
## 2002 "dem_5.1"  "dem_5.2" 
## 2003 "dem_10.1" "dem_10.2"
## 2004 "dem_15.1" "dem_15.2"
## 2005 "dem_20.1" "dem_20.2"
## 2006 "dem_25.1" "dem_25.2"
## 2007 "dem_30.1" "dem_30.2"
## 2008 "dem_35.1" "dem_35.2"
## 2009 "dem_40.1" "dem_40.2"
## 2010 "dem_45.1" "dem_45.2"
\end{verbatim}

RasterArrays are fairly easy to construct: one only needs a stack of the
data and an regular vector/matrix/array including integers. For
instance, the dems object can be recreated from scratch without
problems.

\begin{Shaded}
\begin{Highlighting}[]
\CommentTok{# a stack of rasters}
\NormalTok{stackOfLayers <-}\StringTok{ }\NormalTok{dems}\OperatorTok{@}\NormalTok{stack}
\CommentTok{# an index object}
\NormalTok{ind <-}\StringTok{ }\DecValTok{1}\OperatorTok{:}\DecValTok{10}
\KeywordTok{names}\NormalTok{(ind) <-}\StringTok{ }\NormalTok{letters[}\DecValTok{1}\OperatorTok{:}\DecValTok{10}\NormalTok{]}
\CommentTok{# a RasterArray}
\NormalTok{nra  <-}\StringTok{ }\KeywordTok{RasterArray}\NormalTok{(}\DataTypeTok{index=}\NormalTok{ind, }\DataTypeTok{stack=}\NormalTok{stackOfLayers)}
\NormalTok{nra}
\end{Highlighting}
\end{Shaded}

\begin{verbatim}
## class         : RasterArray 
## RasterLayer properties: 
## - dimensions  : 181, 361  (nrow, ncol)
## - resolution  : 1, 1  (x, y)
## - extent      : -180.5, 180.5, -90.5, 90.5  (xmin, xmax, ymin, ymax)
## - coord. ref. : +proj=longlat +datum=WGS84 +ellps=WGS84 +towgs84=0,0,0 
## Array properties: 
## - dimensions   : 10  (vector)
## - num. layers    : 10
## - proxy:
##         a        b        c        d        e        f        g        h 
##  "dem_0"  "dem_5" "dem_10" "dem_15" "dem_20" "dem_25" "dem_30" "dem_35" 
##        i        j 
## "dem_40" "dem_45"
\end{verbatim}

The attributes of the index object are defining the structure of the
RasterArray. RasterArrays can be created with the combination of
individual RasterLayers (or RasterArrays) using the combine() function.

\begin{Shaded}
\begin{Highlighting}[]
\CommentTok{# one raster}
\NormalTok{r1 <-}\StringTok{ }\KeywordTok{raster}\NormalTok{()}
\KeywordTok{values}\NormalTok{(r1) <-}\StringTok{ }\DecValTok{1}
\CommentTok{# same structure, different value}
\NormalTok{r2 <-}\KeywordTok{raster}\NormalTok{()}
\KeywordTok{values}\NormalTok{(r2) <-}\StringTok{ }\DecValTok{2}
\NormalTok{comb <-}\StringTok{ }\KeywordTok{combine}\NormalTok{(r1, r2)}
\NormalTok{comb}
\end{Highlighting}
\end{Shaded}

\begin{verbatim}
## class         : RasterArray 
## RasterLayer properties: 
## - dimensions  : 180, 360  (nrow, ncol)
## - resolution  : 1, 1  (x, y)
## - extent      : -180, 180, -90, 90  (xmin, xmax, ymin, ymax)
## - coord. ref. : +proj=longlat +datum=WGS84 +ellps=WGS84 +towgs84=0,0,0 
## Array properties: 
## - dimensions   : 2  (vector)
## - num. layers    : 2
## - proxy:
##         r1        r2 
## "layer.1" "layer.2"
\end{verbatim}

Matrix-like RasterArrays can also be created easilly with the, cbind(),
and rbind() functions.

\begin{Shaded}
\begin{Highlighting}[]
\CommentTok{# bind dems to itself}
\KeywordTok{cbind}\NormalTok{(dems, dems)}
\end{Highlighting}
\end{Shaded}

\begin{verbatim}
## class         : RasterArray 
## RasterLayer properties: 
## - dimensions  : 181, 361  (nrow, ncol)
## - resolution  : 1, 1  (x, y)
## - extent      : -180.5, 180.5, -90.5, 90.5  (xmin, xmax, ymin, ymax)
## - coord. ref. : +proj=longlat +datum=WGS84 +ellps=WGS84 +towgs84=0,0,0 
## Array properties: 
## - dimensions  : 10, 2  (nrow, ncol)
## - num. layers    : 20
## - proxy:
##     [,1]       [,2]      
## 0  "dem_0.1"  "dem_0.2" 
## 5  "dem_5.1"  "dem_5.2" 
## 10 "dem_10.1" "dem_10.2"
## 15 "dem_15.1" "dem_15.2"
## 20 "dem_20.1" "dem_20.2"
## 25 "dem_25.1" "dem_25.2"
## 30 "dem_30.1" "dem_30.2"
## 35 "dem_35.1" "dem_35.2"
## 40 "dem_40.1" "dem_40.2"
## 45 "dem_45.1" "dem_45.2"
\end{verbatim}

\hypertarget{rasterarray-attributes-and-function-to-query}{%
\subsection{2.2. RasterArray attributes and function to
query}\label{rasterarray-attributes-and-function-to-query}}

Functions that query and change attributes of the RasterArray resemble
general arrays more than Raster* objects. They are connected to the
index slot of the RasterArray and return values accordingly.

The number of elements represented in the RasterArray can be queried
with the length() function:

\begin{Shaded}
\begin{Highlighting}[]
\KeywordTok{length}\NormalTok{(dems)}
\end{Highlighting}
\end{Shaded}

\begin{verbatim}
## [1] 10
\end{verbatim}

This RasterArray has 10 elements. The number of column and row names can
be queried in a similar way:

\begin{Shaded}
\begin{Highlighting}[]
\KeywordTok{nrow}\NormalTok{(clim)}
\end{Highlighting}
\end{Shaded}

\begin{verbatim}
## [1] 10
\end{verbatim}

\begin{Shaded}
\begin{Highlighting}[]
\KeywordTok{ncol}\NormalTok{(clim)}
\end{Highlighting}
\end{Shaded}

\begin{verbatim}
## [1] 2
\end{verbatim}

These functions are summarized in the dim() funciton. This, however,
unlike the regular dim() method of vectors, return the length of the
RasterArray-vector, rather than just NULL.

\begin{Shaded}
\begin{Highlighting}[]
\KeywordTok{dim}\NormalTok{(dems)}
\end{Highlighting}
\end{Shaded}

\begin{verbatim}
## [1] 10
\end{verbatim}

\begin{Shaded}
\begin{Highlighting}[]
\KeywordTok{dim}\NormalTok{(clim)}
\end{Highlighting}
\end{Shaded}

\begin{verbatim}
## [1] 10  2
\end{verbatim}

The organization of RasterLayers can be greatly facilitated with names.
The names(), colnames(), rownames() and dimnames() functions work the
same way on RasterArrays as if they were arrays of simple numeric,
logical or character values. The names() function returns the names of
individual elements of a vector-like RasterArray.

\begin{Shaded}
\begin{Highlighting}[]
\KeywordTok{names}\NormalTok{(dems)}
\end{Highlighting}
\end{Shaded}

\begin{verbatim}
##  [1] "0"  "5"  "10" "15" "20" "25" "30" "35" "40" "45"
\end{verbatim}

The colnames() and rownames() functions are more relevant for
matrix-like RasterArrays, such as clim.

\begin{Shaded}
\begin{Highlighting}[]
\KeywordTok{colnames}\NormalTok{(clim)}
\end{Highlighting}
\end{Shaded}

\begin{verbatim}
## [1] "bio1"  "bio12"
\end{verbatim}

\begin{Shaded}
\begin{Highlighting}[]
\KeywordTok{rownames}\NormalTok{(clim)}
\end{Highlighting}
\end{Shaded}

\begin{verbatim}
##  [1] "2001" "2002" "2003" "2004" "2005" "2006" "2007" "2008" "2009" "2010"
\end{verbatim}

All name-related methods can be used for replacement as well. For
instance, you can quickly rename the names of the columns of the clim
object this way:

\begin{Shaded}
\begin{Highlighting}[]
\NormalTok{clim2 <-}\StringTok{ }\NormalTok{clim}
\KeywordTok{colnames}\NormalTok{(clim2) <-}\StringTok{ }\KeywordTok{c}\NormalTok{(}\StringTok{"temp"}\NormalTok{, }\StringTok{"prec"}\NormalTok{)}
\NormalTok{clim2}
\end{Highlighting}
\end{Shaded}

\begin{verbatim}
## class         : RasterArray 
## RasterLayer properties: 
## - dimensions  : 181, 361  (nrow, ncol)
## - resolution  : 1, 1  (x, y)
## - extent      : -180.5, 180.5, -90.5, 90.5  (xmin, xmax, ymin, ymax)
## - coord. ref. : +proj=longlat +datum=WGS84 +ellps=WGS84 +towgs84=0,0,0 
## Array properties: 
## - dimensions  : 10, 2  (nrow, ncol)
## - num. layers    : 20
## - proxy:
##       temp       prec      
## 2001 "dem_0.1"  "dem_0.2" 
## 2002 "dem_5.1"  "dem_5.2" 
## 2003 "dem_10.1" "dem_10.2"
## 2004 "dem_15.1" "dem_15.2"
## 2005 "dem_20.1" "dem_20.2"
## 2006 "dem_25.1" "dem_25.2"
## 2007 "dem_30.1" "dem_30.2"
## 2008 "dem_35.1" "dem_35.2"
## 2009 "dem_40.1" "dem_40.2"
## 2010 "dem_45.1" "dem_45.2"
\end{verbatim}

Just as you would do it with normal arrays, the you can query/rewrite
all names with the dimnames() function, that uses a list to store the
names in every dimension.

\begin{Shaded}
\begin{Highlighting}[]
\KeywordTok{dimnames}\NormalTok{(clim2)[[}\DecValTok{1}\NormalTok{]] <-}\StringTok{ }\DecValTok{1}\OperatorTok{:}\DecValTok{10}
\NormalTok{clim2}
\end{Highlighting}
\end{Shaded}

\begin{verbatim}
## class         : RasterArray 
## RasterLayer properties: 
## - dimensions  : 181, 361  (nrow, ncol)
## - resolution  : 1, 1  (x, y)
## - extent      : -180.5, 180.5, -90.5, 90.5  (xmin, xmax, ymin, ymax)
## - coord. ref. : +proj=longlat +datum=WGS84 +ellps=WGS84 +towgs84=0,0,0 
## Array properties: 
## - dimensions  : 10, 2  (nrow, ncol)
## - num. layers    : 20
## - proxy:
##     temp       prec      
## 1  "dem_0.1"  "dem_0.2" 
## 2  "dem_5.1"  "dem_5.2" 
## 3  "dem_10.1" "dem_10.2"
## 4  "dem_15.1" "dem_15.2"
## 5  "dem_20.1" "dem_20.2"
## 6  "dem_25.1" "dem_25.2"
## 7  "dem_30.1" "dem_30.2"
## 8  "dem_35.1" "dem_35.2"
## 9  "dem_40.1" "dem_40.2"
## 10 "dem_45.1" "dem_45.2"
\end{verbatim}

Besides the names of the eements in the RasterArray, every layer has its
own name in the stack. These can be accessed with layers() function:

\begin{Shaded}
\begin{Highlighting}[]
\KeywordTok{layers}\NormalTok{(clim)}
\end{Highlighting}
\end{Shaded}

\begin{verbatim}
##  [1] "dem_0.1"  "dem_5.1"  "dem_10.1" "dem_15.1" "dem_20.1" "dem_25.1"
##  [7] "dem_30.1" "dem_35.1" "dem_40.1" "dem_45.1" "dem_0.2"  "dem_5.2" 
## [13] "dem_10.2" "dem_15.2" "dem_20.2" "dem_25.2" "dem_30.2" "dem_35.2"
## [19] "dem_40.2" "dem_45.2"
\end{verbatim}

The total number of cells in the RasterLayer or the entire stack can
accessed with the ncell() and nvalues() functions, respectively.

\begin{Shaded}
\begin{Highlighting}[]
\KeywordTok{ncell}\NormalTok{(dems)}
\end{Highlighting}
\end{Shaded}

\begin{verbatim}
## [1] 65341
\end{verbatim}

\begin{Shaded}
\begin{Highlighting}[]
\KeywordTok{nvalues}\NormalTok{(dems)}
\end{Highlighting}
\end{Shaded}

\begin{verbatim}
## [1] 653410
\end{verbatim}

\hypertarget{subsetting-and-replacement}{%
\subsection{2.3. Subsetting and
replacement}\label{subsetting-and-replacement}}

Facilitating the accession of items is the primary purpose of
RasterArrays. These either focus on the layers (stack items, double
bracket operator ``{[}{[}'') or the elements of the RasterArrray (single
bracket operator ``{[}'').

\hypertarget{layer-selection---double-bracket}{%
\subsubsection{2.3.1 Layer selection - Double bracket
{[}{[}}\label{layer-selection---double-bracket}}

This form of subsetting and replacement are inherited from the
RasterStack class. Individual layers can be accessed directly from the
stack using either the position index, the name of the layer or the
logical value pointing to the position. Whichever is used, the
RasterArray wrapper omitted and output will a RasterLayer or RasterStack
class object.

A single layer can be accessed using its name, regardless of its
position in the RasterArray. This can be visulized either with the
default plot() or the more general mapplot() funciton.

\begin{Shaded}
\begin{Highlighting}[]
\NormalTok{one <-}\StringTok{ }\NormalTok{dems[[}\StringTok{"dem_45"}\NormalTok{]]}
\KeywordTok{mapplot}\NormalTok{(one, }\DataTypeTok{col=}\StringTok{"earth"}\NormalTok{)}
\end{Highlighting}
\end{Shaded}

\includegraphics{chronos_files/figure-latex/single-1.pdf}

Returning a single RasterArray. Multiple elements will be the format of
a stack:

\begin{Shaded}
\begin{Highlighting}[]
\NormalTok{dems[[}\KeywordTok{c}\NormalTok{(}\DecValTok{1}\NormalTok{,}\DecValTok{2}\NormalTok{)]]}
\end{Highlighting}
\end{Shaded}

\begin{verbatim}
## class      : RasterStack 
## dimensions : 181, 361, 65341, 2  (nrow, ncol, ncell, nlayers)
## resolution : 1, 1  (x, y)
## extent     : -180.5, 180.5, -90.5, 90.5  (xmin, xmax, ymin, ymax)
## crs        : +proj=longlat +datum=WGS84 +ellps=WGS84 +towgs84=0,0,0 
## names      : dem_0, dem_5 
## min values : -9000, -7000 
## max values :  6000,  6300
\end{verbatim}

Which are the first two RasterLayers in the stack of the RasterArray.

Double brackets can also be used for replacements, but as this has no
effect on the structure of the array, changes implemented with this
method are more difficult to trace. For instance,

\begin{Shaded}
\begin{Highlighting}[]
\CommentTok{# copy}
\NormalTok{dem2 <-}\StringTok{ }\NormalTok{dems}
\NormalTok{dem2[[}\StringTok{"dem_0"}\NormalTok{]] <-}\StringTok{ }\NormalTok{dem2[[}\StringTok{"dem_5"}\NormalTok{]]}
\end{Highlighting}
\end{Shaded}

will rewrite the values in the first element of dem2, but that will not
be evident in the RasterArray's structure.

\begin{Shaded}
\begin{Highlighting}[]
\CommentTok{# but these two are now the same}
\NormalTok{dem2[[}\DecValTok{1}\NormalTok{]]}
\end{Highlighting}
\end{Shaded}

\begin{verbatim}
## class      : RasterLayer 
## dimensions : 181, 361, 65341  (nrow, ncol, ncell)
## resolution : 1, 1  (x, y)
## extent     : -180.5, 180.5, -90.5, 90.5  (xmin, xmax, ymin, ymax)
## crs        : +proj=longlat +datum=WGS84 +ellps=WGS84 +towgs84=0,0,0 
## source     : memory
## names      : dem_0 
## values     : -7000, 6300  (min, max)
## zvar       : z
\end{verbatim}

\begin{Shaded}
\begin{Highlighting}[]
\NormalTok{dem2[[}\DecValTok{2}\NormalTok{]]}
\end{Highlighting}
\end{Shaded}

\begin{verbatim}
## class      : RasterLayer 
## dimensions : 181, 361, 65341  (nrow, ncol, ncell)
## resolution : 1, 1  (x, y)
## extent     : -180.5, 180.5, -90.5, 90.5  (xmin, xmax, ymin, ymax)
## crs        : +proj=longlat +datum=WGS84 +ellps=WGS84 +towgs84=0,0,0 
## source     : memory
## names      : dem_5 
## values     : -7000, 6300  (min, max)
## zvar       : z
\end{verbatim}

\hypertarget{single-bracket}{%
\subsubsection{2.3.2 Single bracket}\label{single-bracket}}

Features offered by the double bracket (``{[}{[}'') operator are
virtually identical with those of RasterStacks. The true utility of
RasterArrays become evident with simple array-type subsetting.

Unike Raster* objects of the raster package, single brackets will get
and replace items from the RasterArray as if they were simple arrays.
For example, single elements of the DEMs can be selected with the age of
the DEM, passed as a character subscript.

\begin{Shaded}
\begin{Highlighting}[]
\NormalTok{dems[}\StringTok{"30"}\NormalTok{]}
\end{Highlighting}
\end{Shaded}

\begin{verbatim}
## class      : RasterLayer 
## dimensions : 181, 361, 65341  (nrow, ncol, ncell)
## resolution : 1, 1  (x, y)
## extent     : -180.5, 180.5, -90.5, 90.5  (xmin, xmax, ymin, ymax)
## crs        : +proj=longlat +datum=WGS84 +ellps=WGS84 +towgs84=0,0,0 
## source     : memory
## names      : dem_30 
## values     : -8000, 10200  (min, max)
## zvar       : z
\end{verbatim}

returning the 30Ma RasterLayer. By default the RasterArray container is
dropped, but it can be conserved, if the drop argument is set to FALSE.

\begin{Shaded}
\begin{Highlighting}[]
\NormalTok{dem30 <-}\StringTok{ }\NormalTok{dems[}\StringTok{"30"}\NormalTok{, drop=}\OtherTok{FALSE}\NormalTok{]}
\KeywordTok{class}\NormalTok{(dem30)}
\end{Highlighting}
\end{Shaded}

\begin{verbatim}
## [1] "RasterArray"
## attr(,"package")
## [1] "chronosphere"
\end{verbatim}

Beyond bounds accessing is valid for single dimensional RasterArrays
(vector-like ones):

\begin{Shaded}
\begin{Highlighting}[]
\NormalTok{dems[}\DecValTok{4}\OperatorTok{:}\DecValTok{12}\NormalTok{]}
\end{Highlighting}
\end{Shaded}

\begin{verbatim}
## class         : RasterArray 
## RasterLayer properties: 
## - dimensions  : 181, 361  (nrow, ncol)
## - resolution  : 1, 1  (x, y)
## - extent      : -180.5, 180.5, -90.5, 90.5  (xmin, xmax, ymin, ymax)
## - coord. ref. : +proj=longlat +datum=WGS84 +ellps=WGS84 +towgs84=0,0,0 
## Array properties: 
## - dimensions   : 9  (vector)
## - num. layers    : 7
## - proxy:
##        15       20       25       30       35       40       45     <NA> 
## "dem_15" "dem_20" "dem_25" "dem_30" "dem_35" "dem_40" "dem_45"       NA 
##     <NA> 
##       NA
\end{verbatim}

Missing values are legitimate parts of RasterArrays. These gaps in the
data are not represented in the stacks, but only in the index slots of
the RasterArrays. They can be inserted or added into the layers.

\begin{Shaded}
\begin{Highlighting}[]
\NormalTok{demna <-}\StringTok{ }\NormalTok{dems}
\NormalTok{demna[}\DecValTok{3}\NormalTok{] <-}\StringTok{ }\OtherTok{NA}
\end{Highlighting}
\end{Shaded}

Multidimensional subscripts work in a similar fashion. If a single layer
is desired from the RasterArray, that can be accessed using the names of
the margins.

\begin{Shaded}
\begin{Highlighting}[]
\CommentTok{# character is necessary, as the row named "2003" is necessary}
\NormalTok{one <-}\StringTok{ }\NormalTok{clim[}\StringTok{"2003"}\NormalTok{, }\StringTok{"bio1"}\NormalTok{]}
\KeywordTok{mapplot}\NormalTok{(one)}
\end{Highlighting}
\end{Shaded}

\includegraphics{chronos_files/figure-latex/cellsbu-1.pdf}

similarly to entire rows,

\begin{Shaded}
\begin{Highlighting}[]
\NormalTok{clim[}\StringTok{"2005"}\NormalTok{, ]}
\end{Highlighting}
\end{Shaded}

\begin{verbatim}
## class         : RasterArray 
## RasterLayer properties: 
## - dimensions  : 181, 361  (nrow, ncol)
## - resolution  : 1, 1  (x, y)
## - extent      : -180.5, 180.5, -90.5, 90.5  (xmin, xmax, ymin, ymax)
## - coord. ref. : +proj=longlat +datum=WGS84 +ellps=WGS84 +towgs84=0,0,0 
## Array properties: 
## - dimensions   : 2  (vector)
## - num. layers    : 2
## - proxy:
##        bio1      bio12 
## "dem_20.1" "dem_20.2"
\end{verbatim}

or columns

\begin{Shaded}
\begin{Highlighting}[]
\NormalTok{clim[,}\StringTok{"bio12"}\NormalTok{]}
\end{Highlighting}
\end{Shaded}

\begin{verbatim}
## class         : RasterArray 
## RasterLayer properties: 
## - dimensions  : 181, 361  (nrow, ncol)
## - resolution  : 1, 1  (x, y)
## - extent      : -180.5, 180.5, -90.5, 90.5  (xmin, xmax, ymin, ymax)
## - coord. ref. : +proj=longlat +datum=WGS84 +ellps=WGS84 +towgs84=0,0,0 
## Array properties: 
## - dimensions   : 10  (vector)
## - num. layers    : 10
## - proxy:
##        2001       2002       2003       2004       2005       2006 
##  "dem_0.2"  "dem_5.2" "dem_10.2" "dem_15.2" "dem_20.2" "dem_25.2" 
##       2007       2008       2009       2010 
## "dem_30.2" "dem_35.2" "dem_40.2" "dem_45.2"
\end{verbatim}

\hypertarget{inherited-from-raster}{%
\subsubsection{2.4 Inherited from Raster*}\label{inherited-from-raster}}

As the spatial information is contained entirely in the RasterStacks, a
number methods are practically inherited from RasterStack class. For
instance, all RasterLayer of the RasterArray can be cropped in a single
line of code.

\begin{Shaded}
\begin{Highlighting}[]
\CommentTok{# crop to Australia}
\NormalTok{ext <-}\StringTok{ }\KeywordTok{extent}\NormalTok{(}\KeywordTok{c}\NormalTok{(                }
  \DataTypeTok{xmin =} \FloatTok{106.58}\NormalTok{,}
  \DataTypeTok{xmax =} \FloatTok{157.82}\NormalTok{,}
  \DataTypeTok{ymin =} \FloatTok{-45.23}\NormalTok{,}
  \DataTypeTok{ymax =} \FloatTok{1.14} 
\NormalTok{)) }

\CommentTok{# cropping all DEMS (Australia drifted in)}
\NormalTok{au<-}\StringTok{ }\KeywordTok{crop}\NormalTok{(dems, ext)}

\CommentTok{# select the first element}
\KeywordTok{mapplot}\NormalTok{(au[}\DecValTok{1}\NormalTok{], }\DataTypeTok{col=}\StringTok{"earth"}\NormalTok{)}
\end{Highlighting}
\end{Shaded}

\includegraphics{chronos_files/figure-latex/crop-1.pdf}

Other functions such as aggregation or resampling works just the same.

\begin{Shaded}
\begin{Highlighting}[]
\NormalTok{template <-}\StringTok{ }\KeywordTok{raster}\NormalTok{(}\DataTypeTok{res=}\DecValTok{5}\NormalTok{)}

\CommentTok{# resample all DEMS}
\NormalTok{coarse <-}\StringTok{ }\KeywordTok{resample}\NormalTok{(dems, template)}

\CommentTok{# plot an elemnt}
\KeywordTok{mapplot}\NormalTok{(coarse[}\StringTok{"45"}\NormalTok{], }\DataTypeTok{col=}\StringTok{"earth"}\NormalTok{)}
\end{Highlighting}
\end{Shaded}

\includegraphics{chronos_files/figure-latex/resam-1.pdf}

\hypertarget{plotting}{%
\subsection{3. Plotting}\label{plotting}}

mapplot() - RasterLayer -RasterArray{[}1 dimension{]} - RasterArray{[}2
dimension{]}

each chunk has to have a unique name (after r) otherwise the
vignettebuilder will give an error.

The echo field sets wether the code is visible. eval triggers whether it
is evaluated.


\end{document}
